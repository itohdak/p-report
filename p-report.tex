\documentclass[onecolumn]{preport}
\usepackage[dvipdfmx]{graphicx}
\graphicspath{{figs/}}

\usepackage{amsmath, amssymb}
\def\vector#1{\mbox{\boldmath $#1$}}

\title{先端データ解析論 第2回レポート}
\author{48-186619 伊藤秀朗}

\begin{document}

\pagestyle{empty}
\maketitle
\thispagestyle{empty}
\sloppy

\section{宿題2}
\def\Phii{\Phi_i}
\def\phii{\phi_i}
\def\xv{\vector{x}}
\def\thetav{\vector{\theta}}
\def\thetil{\vector{\widehat\theta}}
線形モデル \(f_{\thetav}(\xv) = \sum_{j=1}^{b}\theta_j\phi_j(\xv)\) を用いたL2正則化回帰に対する最適解
\begin{align}
  \thetil &= (\Phi^\top \Phi + \lambda I)^{-1} \Phi^\top \vector{y}
\end{align}
を用いて, \((\xv_i, y_i)\) を抜いて学習したパラメータ \(\thetil_i\) は,
\begin{align}
  %% \Phii &= \Phi - \vector{t}_i \phi_i^\top \\
  %% \Phii^\top \Phii &= (\Phi - \vector{t}_i \phi_i^\top)^\top (\Phi - \vector{t}_i \phi_i^\top) \\
  %% &= (\Phi^\top - \phi_i \vector{t}_i^\top) (\Phi - \vector{t}_i \phi_i^\top) \\
  %% &= \Phi^\top \Phi - \Phi^\top \vector{t}_i \phi_i^\top - \phi_i \vector{t}_i^\top \Phi + \phi_i \vector{t}_i^\top \vector{t}_i \phi_i^\top \\
  %% &= \Phi^\top \Phi - \phi_i \phi_i^\top - \phi_i \phi_i^\top + \phi_i \phi_i^\top \\
  %% &= \Phi^\top \Phi - \phi_i \phi_i^\top \\
  %% \vector{t}_i &= (0, \cdots, 0, 1, 0, \cdots, 0)^\top \\
  %% \thetil_i &= (\Phi^\top \Phi + \lambda I - \phi_i \phi_i^\top)^{-1} (\Phi - \vector{t}_i \phi_i^\top)^\top (\vector{y} - y_i \vector{t})
  \thetil_i &= (\Phi^\top \Phi + \lambda I - \phi_i \phi_i^\top)^{-1} (\Phi^\top \vector{y} - y_i \phi_i)
\end{align}
と書ける.
さらに,\(U = \Phi^\top \Phi + \lambda I, \ \ \alpha_i = \phii^\top U^{-1} \phi_i\) を用いて,
\begin{align}
  %% &= (U - \phi_i \phi_i^\top)^{-1} (\Phi^\top - \phi_i \vector{t}_i^\top) (\vector{y} - y_i \vector{t}) \\
  %% &= (U^{-1} + \frac{U^{-1} \phi_i \phi_i^\top U^{-1}}{1 - \phi_i^\top U^{-1} \phi_i}) (\Phi^\top \vector{y} - y_i \Phi^\top \vector{t} - \phi_i \vector{t}_i^\top \vector{y} + y_i \phi_i \vector{t}_i^\top \vector{t}) \\
  %% &= (U^{-1} + \frac{U^{-1} \phi_i \phi_i^\top U^{-1}}{1 - \phi_i^\top U^{-1} \phi_i}) (\Phi^\top \vector{y} - y_i \phi_i - \phi_i y_i + y_i \phi_i) \\
  %% &= (U^{-1} + \frac{U^{-1} \phi_i \phi_i^\top U^{-1}}{1 - \phi_i^\top U^{-1} \phi_i}) (\Phi^\top \vector{y} - y_i \phi_i) \\
  \thetil_i &= (U - \phi_i \phi_i^\top)^{-1} (\Phi^\top \vector{y} - y_i \phi_i) \\
  &= \left(U^{-1} + \frac{U^{-1} \phi_i \phi_i^\top U^{-1}}{1 - \phi_i^\top U^{-1} \phi_i}\right) (\Phi^\top \vector{y} - y_i \phi_i) \ \ (\because Sherman\mathchar`-Morrison formula)\\
  &= \thetil - U^{-1} \phii y_i + \frac{U^{-1} \phii \phii^\top \thetil}{1 - \alpha_i} - \frac{U^{-1} \phii y_i \alpha_i}{1 - \alpha_i} \\
  &= \thetil + \frac{U^{-1} \phi_i \phi_i^\top \thetil - U^{-1} \phi_i y_i}{1 - \alpha_i}
  %% \alpha_i &= \phi_i^\top U^{-1} \phi_i \\ \\
  %%
\end{align}
と変形できる.
よって,このときの \(y_i\) と予測値 \(\phii^\top \thetil_i\) との誤差は,
\begin{align}
  \phi_i^\top \thetil_i - y_i &= \phi_i^\top \left( \thetil + \frac{U^{-1} \phi_i \phi_i^\top \thetil - U^{-1} \phi_i y_i}{1 - \alpha_i} \right) - y_i \\
  &= \frac{1}{1 - \alpha_i} \left[ \left\{ (1 - \alpha_i) + \alpha_i \right\} \phi_i^\top \thetil - \left\{ \alpha_i + (1 - \alpha_i) \right\} y_i \right] \\
  &= \frac{1}{1 - \alpha_i} (\phi_i^\top \thetil - y_i)
\end{align}
と書け,これらの \(1\) から \(n\) の和を取ることで,一つ抜き交差確認法(Leave-One-Out Cross-Validation)における二乗誤差 \(E_{LOOCV}\)は,
\begin{align}
  E_{LOOCV} &= \frac{1}{n} \sum_{i=1}^{n} \left[ \frac{1}{1 - \alpha_i} (\phi_i^\top \thetil - y_i) \right]^2 \\
  &= \frac{1}{n} \| \widetilde{H}^{-1} \left( \Phi^\top U^{-1} \Phi \vector{y} - \vector{y} \right) \|^2 \\
  &= \frac{1}{n} \| \widetilde{H}^{-1} H \vector{y} \|^2
\end{align}
と計算できる.

%% \bibliographystyle{junsrt}
%% \bibliography{p-report}

\end{document}

