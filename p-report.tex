\documentclass[onecolumn]{preport}
\usepackage[dvipdfmx]{graphicx}
\graphicspath{{figs/}}

\usepackage{listings}

\title{クラウド基盤ソフトウェア レポート課題2}
\author{48-186619 創造情報学専攻 伊藤秀朗}

\begin{document}

\pagestyle{empty}
\maketitle
\thispagestyle{empty}
\sloppy

\section{目的}
事前に機械語にコンパイルされる言語,JVM上で動作する言語,動的型言語の3種類で,同等の処理を行うプログラムを作成し,その実行速度を比較する.言語としてはそれぞれ,C++,Java,Pythonを用いた.

\section{処理内容}
今回3種類の言語にて行わせる処理としては行列積を採用した.具体的には,100x100の行列(配列)を2つ用意し,その行列積を計算する.同じ操作を10回繰り返し,実行時間の変化を見た.最後に,10回のうち最小の実行時間を出力する.

\section{実行環境}
\begin{table}[tbh]
  \begin{tabular}{ll}
    OS: & Linux 4.13.0-45-generic\\
    CPU: & Intel(R) Core(TM) i7-6600U CPU @ 2.60GHz\\
    Total memory: & 24GB\\
    Compiler version: & g++ 5.4.0\\
    Compile option: & -O3\\
    JVM version: & JDK 1.8.0\_171\\
    Python version: & Python 2.7.12\\
  \end{tabular}
\end{table}


%% 本稿はプログレスレポートのテンプレートである\cite{Sakai}.

%% 本稿における「、」や「。」は、\verb|make pub|を実行することで、「,」や「.」に変更される。

%% 図は\figref{nowprinting}や\tabref{sample}として参照する.

%% \begin{figure}[tbh]
%%  \begin{center}
%%   \begin{minipage}{0.3\columnwidth}
%%    \includegraphics[width=\columnwidth]{nowprinting.eps}
%%    \caption{eps図の参考例}
%%   \end{minipage}
%%   \hspace{0.15\columnwidth}
%%   \begin{minipage}{0.3\columnwidth}
%%    \includegraphics[width=\columnwidth]{dj.jpg}
%%    \caption{jpg図の参考例}
%%   \end{minipage}
%%   \label{figure:nowprinting}
%%  \end{center}
%% \end{figure}

%% \begin{table}[tbh]
%%  \begin{center}
%%   \begin{tabular}{|l|r|} \hline
%%   A1 & B1 \\
%%   A2 & B2 \\ \hline
%%   \end{tabular}
%%   \caption{図の参考例}
%%   \label{table:sample}
%%  \end{center}
%% \end{table}

\section{ソースコードと実行方法}
\subsection{C++}
\begin{lstlisting}[basicstyle=\ttfamily\footnotesize, frame=single, caption=mat.cpp]
#include <iostream>
#include <iomanip>
#include <stdlib.h>
#include <time.h>

#define N 100
double A[N][N], B[N][N], C[N][N];

int main(){
  /* initialize */
  srand((unsigned int)time(NULL));
  for(int i=0; i<N; i++){
    for(int j=0; j<N; j++){
      A[i][j] = 1.0 * rand() / RAND_MAX;
      B[i][j] = 1.0 * rand() / RAND_MAX;
      C[i][j] = 0.0;
    }
  }
  std::cout << ``initialization finish'' << std::endl;

  double best = 1000000;
  for(int n=0; n<10; n++){
    clock_t start = clock();
    for(int i=0; i<N; i++)
      for(int j=0; j<N; j++)
      for(int k=0; k<N; k++)
        C[i][j] += A[i][k] * B[k][j];
    clock_t end = clock();
    if(best > 1e+3 * (double)(end-start) / CLOCKS_PER_SEC)
      best = 1e+3 * (double)(end-start) / CLOCKS_PER_SEC;
    std::cout << std::fixed
          << std::setprecision(2) << 1e+3 * (double)(end-start) / CLOCKS_PER_SEC << `` msec.'' << std::endl;

    /* initialize */
    for(int i=0; i<N; i++)
      for(int j=0; j<N; j++)
      C[i][j] = 0.0;
  }
  std::cout << std::fixed
      << std::setprecision(2) << best << `` msec.'' << std::endl;

  return 0;
}
\end{lstlisting}
\begin{lstlisting}[basicstyle=\ttfamily\footnotesize, frame=single]
$ g++ mat.cpp -O3
$ ./a.out
\end{lstlisting}

\subsection{Java}
\begin{lstlisting}[basicstyle=\ttfamily\footnotesize, frame=single, caption=mat.java]
public class mat {
    public static void main(String[] args){
int N = 100;
double[][] A = new double[N][N];
double[][] B = new double[N][N];
double[][] C = new double[N][N];
/* initialize */
for(int i=0; i<N; i++){
    for(int j=0; j<N; j++){
A[i][j] = Math.random();
B[i][j] = Math.random();
C[i][j] = 0.0;
    }
}
System.out.println(``initialization finish'');

double best = 1000000;
for(int n=0; n<10; n++){
    long start = System.nanoTime();
    for(int i=0; i<N; i++)
for(int j=0; j<N; j++)
    for(int k=0; k<N; k++)
C[i][j] += A[i][k] * B[k][j];
    long end = System.nanoTime();
    if(best > (end-start)/1e+6)
best = (end-start)/1e+6;
    System.out.println(String.format(``%6.2f'', (end-start)/1e+6) + `` msec.'');

    /* initialize */
    for(int i=0; i<N; i++)
for(int j=0; j<N; j++)
    C[i][j] = 0.0;
}
System.out.println(String.format(``%6.2f'', best) + `` msec.'');
    }
}
\end{lstlisting}
\begin{lstlisting}[basicstyle=\ttfamily\footnotesize, frame=single]
$ javac mat.java
$ java mat
\end{lstlisting}

\subsection{Python}
\begin{lstlisting}[basicstyle=\ttfamily\footnotesize, frame=single, caption=mat.py]
import random
import time

if __name__ == '__main__':
    N = 100
    # initialize
    A = [[random.uniform(0.0, 1.0) for i in range(N)] for j in range(N)]
    B = [[random.uniform(0.0, 1.0) for i in range(N)] for j in range(N)]
    C = [[0.0 for i in range(N)] for j in range(N)]
    print(``initialization finish'')

    best = 1000000
    for n in range(10):
        start = time.time()
        for i in range(N):
            for j in range(N):
                for k in range(N):
                    C[i][j] += A[i][k] * B[k][j]
        end = time.time()
        if best > (end-start)*1e+3:
            best = (end-start)*1e+3
        print(``{0:6.2f} msec.''.format((end-start)*1e+3))

        # initialize
        C = [[0.0 for i in range(N)] for j in range(N)]
    print(``{0:6.2f} msec.''.format(best))
\end{lstlisting}
\begin{lstlisting}[basicstyle=\ttfamily\footnotesize, frame=single]
$ python mat.py
\end{lstlisting}

\section{結果}
以下は,行列の大きさを表すNがN=100のときの実行結果である.
動的型言語であるPythonは,他の2つに比べて,実行時間が100倍以上大きいことがわかる.
C++とJavaとの比較においては,N=100のときには,最小実行時間にあまり差は見られない.最大実行時間はJavaのほうが大きく出ている.ここには示していないが,Javaでの複数回の試行において,高い確率で1回目と3回目で,他と比較して大きい実行時間を記録しており,JVMの実行の仕組みが関係していそうな特徴がある.
\begin{lstlisting}[basicstyle=\ttfamily\footnotesize, frame=single]
$ ./a.out
initialization finish
5.56 msec.
5.47 msec.
5.46 msec.
4.30 msec.
3.29 msec.
2.55 msec.
2.42 msec.
1.80 msec.
1.80 msec.
1.69 msec.
1.69 msec.
\end{lstlisting}

\begin{lstlisting}[basicstyle=\ttfamily\footnotesize, frame=single]
$ java mat
initialization finish
 10.71 msec.
  1.99 msec.
 30.92 msec.
  1.32 msec.
  1.19 msec.
  1.21 msec.
  1.25 msec.
  1.47 msec.
  1.98 msec.
  1.99 msec.
  1.19 msec.
\end{lstlisting}

\begin{lstlisting}[basicstyle=\ttfamily\footnotesize, frame=single]
$ python mat.py
initialization finish
208.97 msec.
206.84 msec.
208.16 msec.
227.97 msec.
220.70 msec.
212.56 msec.
210.64 msec.
208.89 msec.
214.86 msec.
212.89 msec.
206.84 msec.
\end{lstlisting}

N=1000に変更して,実行してみた結果を以下に示す.N=1000では,C++のほうが実行時間が小さくなった.

\begin{lstlisting}[basicstyle=\ttfamily\footnotesize, frame=single]
$ ./a.out
initialization finish
1852.78 msec.
1827.35 msec.
1836.57 msec.
1879.23 msec.
1845.83 msec.
1819.49 msec.
1817.25 msec.
1844.97 msec.
1859.28 msec.
1843.92 msec.
1817.25 msec.
\end{lstlisting}

\begin{lstlisting}[basicstyle=\ttfamily\footnotesize, frame=single]
$ java mat
initialization finish
6026.17 msec.
6064.88 msec.
6061.86 msec.
6073.92 msec.
6050.69 msec.
6051.95 msec.
6054.98 msec.
6034.34 msec.
6019.58 msec.
6063.23 msec.
6019.58 msec.
\end{lstlisting}
\bibliographystyle{junsrt}
%% \bibliography{p-report}

\end{document}

