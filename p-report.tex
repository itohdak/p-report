\documentclass[onecolumn]{preport}
\usepackage[dvipdfmx]{graphicx}
\graphicspath{{figs/}}

\title{画像処理論 最終レポート}
\author{情報理工学系研究科 創造情報学専攻\\
48-186619 伊藤秀朗}

\begin{document}

\pagestyle{empty}
\maketitle
\thispagestyle{empty}
\sloppy

\Large

\section{はじめに}

本レポートでは,候補論文3本のうち,``Building Rome in a Day''~\cite{Rome}を選択し,その内容についてまとめ,そして議論する.

\section{参考論文における目的と手法}

参考論文は,写真共有サイトFlickrで検索して得られる特定の都市の画像について,その膨大な数の画像の中から同じ事物について撮影されている画像群を特徴点抽出を用いて探し出し,そのそれぞれから復元された3次元点群を連ねていくことで,3次元の都市のモデルを再構築しようという主旨の内容である.これらの5万〜数10万の画像データから1つの巨大な都市を3次元的に復元すること自体も驚きであるが,それらの膨大な量の計算を複数のPCを用いてたった1日で完了するように計算の手順等の最適化に工夫が凝らされている.その詳しい中身に関して,以下で見ていく.

大量の画像を用いたある都市の3次元再構築は主に以下の手順で行われる.

\begin{enumerate}
  \item 画像間におけるマッチングポイントの探索
  \item 構図の近い(撮影地点の近い)画像系列を結ぶグラフ(match graph)の作成
  \item SfM (Struction from Motion)を用いた撮影地点の3次元推定
  \item 被写体の3次元再構築
\end{enumerate}

まず,何万の画像の中から,画像同士を結びつける頼りとなる出現頻度の高い特徴点を導き出す必要がある.本論文ではSIFT特徴量~\cite{SIFT}とANN (Approximate Nearest Neighbor; 近似最近傍探索)~\cite{ANN}を組み合わせた時間計算量の観点で効率的な方法でこの手順を達成している.また,被写体に剛体を想定し,RANSAC~\cite{RANSAC}を併用することで,マッチングの剪定を行っている.

しかし,本論文では大規模なデータを対象に操作を行うことを前提としており,すべての画像データを1対1で比較した際に想定される\(O(n^2)\)の時間計算量はボトルネックとなる.そこで本論文の1つの特徴であるマルチラウンドな手法が提案されている.これは,第1段階としてvisual wordsを用いた類似性評価を行い,類似度の高い上位20の画像間でスパースなグラフを作成し(whole image similarity),第2段階でクエリを用いたグラフの拡大を行う(query expansion)ものである.

その後,グラフでペアになった画像群をもとに,画像上で共通な1点を算出する.それら共通の点をもとに,SfMの手法を用いた最適化計算によりそれぞれの画像が撮影されたカメラの三次元位置を逆算する.カメラで撮影された画像上の点が,以下の式で表されたとする.
\begin{equation}
  x_{ij} = f_j\Pi(R_j(X_i-c_j))
\end{equation}
SfM (Struction from Motion)とは,そのとき,複数の画像上で撮影された同一の点から以下の式により,最適な3次元座標およびカメラパラメータを推定するものである.
\begin{equation}
  \rm{arg}\min_{X_i,R_j,c_j,f_j}\left(\sum_{i-j}\|x_{ij}-f_j\Pi(R_j(X_i-c_j)\|^2\right)
\end{equation}

ここでも,計算時間削減のため工夫が施されており,ある都市の名所の画像はたいてい同じような画角で取られることが多いという仮定のもと,これまでに算出した,共通特徴点が見つかった大量の画像のうち,そのいくつかをこの最適化計算に用い,そこから導出された特徴点の3次元位置からその他の画像が撮影されたカメラのパラメータを1ステップで推定するというものである.それにより計算が重くなる最適化計算の回数を最小限に減らし,速度向上に貢献している.

最後に,よりデンスな3次元復元を達成するため,SfMで得られた結果と対応する画像を用いて,MVS (multiview stereo) algorithmによる3次元点推定がなされている.ここでもメモリの使い過ぎを防ぐため,より少ないクラスタを処理の対象にするという最適化が施されている.

本論文は,上記でまとめた手法の他に,これらの過程の分散処理にも特徴があり,計算ノード間の通信をできる限り廃した分散処理計画により,高速な一連の操作が可能になっている.

\section{考察}

本論文は,大量の画像データをもとに都市の3次元再構築を実現するものであり,その成果は驚くべきものである.しかし,いくつか制約は存在する.まず1つに天井や床などのサンプル数の少ない,または,多くの場合に見えない箇所の点群は再構成できない点がある.また,本論文では特徴点のマッチングにより一連のグラフを作成することによって3次元復元がなされるが,撮影頻度が少なく,データの少ない区画などが存在すると,そのグラフとして孤立したクラスタを複数生成される可能性があるという点である.後者の解決法としては,画像に添付されているGPSなどの位置情報データをもとに,各クラスタの相対的な位置関係を最小二乗法的に最適化するというものが考えられる.

さらには,例えば開発途上国などその都市の景観が著しく変化する,もしくは,幅広い年代の画像が入り混じっているような場合には,その変化に対応できない可能性も考えられる.自然景観などの特徴量の比較的少ない単調な画像群などでもマッチングは難しそうである.

近年では,画像の編集技術が向上し,その影響で画像が実際に撮影された状態のものでない加工されたものであることも多くなっており,特徴点はマッチングされているにもかかわらず,点の対応が歪んでいたりする可能性も考えられる.画像共有サイトなどの大規模な画像データを用いる際は,それを考慮に入れた上で取捨選択を行う必要がある.

\section{おわりに}

本レポートでは,課題論文の1つである``Building Rome in a Day''~\cite{Rome}を読み,その内容をまとめたのち,その手法に含まれる制約について考察した.
本レポートは画像処理論の最終レポートであるが,その講義においては画像処理やカメラの基礎となる重要テーマについて,細かい数式を交えた丁寧な解説が行われた.近年注目されているディープラーニングなどの機械学習が生じる以前から研究されてきた理論的な分野であり,現在でも非常に重要な技術である.それらの事項を改めて一から見直し,また,知ることができ,とても有意義な講義であった.

\large
\bibliographystyle{junsrt}
\bibliography{p-report}

\end{document}

