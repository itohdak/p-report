\documentclass[onecolumn]{preport}
\usepackage[dvipdfmx]{graphicx}
\graphicspath{{figs/}}

\usepackage{amsmath,amssymb}
\def\vector#1{\mbox{\boldmath $#1$}}

\title{先端データ解析論 第5回レポート}
\author{48-186619 伊藤秀朗}

\begin{document}

\pagestyle{empty}
\maketitle
\thispagestyle{empty}
\sloppy

\section{宿題1}

\def\Xt{X^\top}
\def\x{\vector{x}}
\def\xt{\vector{x}^\top}
\def\y{\vector{y}}
\def\thv{\vector{\theta}}
\def\muv{\vector{\mu}}
\def\mumin{\vector{\mu}_{-}}
\def\muplus{\vector{\mu}_{+}}
\def\nt{n_t}
\def\nmin{n_{-}}
\def\nplus{n_{+}}

最小二乗最適化により得られる式
\begin{align}
  \Xt X \thv = \Xt \y \label{equation:problem}
\end{align}
について考える.右辺の\(\Xt \y\)について,
\begin{align}
  \nmin \mumin + \nplus \muplus &= n \mu \\
  &= \vector{0} \\
  \therefore \mumin &= - \frac{\nplus}{\nmin} \muplus
\end{align}
より,
\begin{align}
  \Xt \y &= \sum_{i=1}^n{y_i \x_i} \\
  &= \sum_{y_i=1}{\x_{i, +}} - \sum_{y_i=-1}{\x_{i, -}} \\
  &= \nplus \muplus - \nmin \mumin \\
  &= \nplus \muplus + \nmin \frac{\nplus}{\nmin} \muplus \\
  \therefore \Xt \y &= 2 \nplus \muplus
\end{align}
となる.なお,
\begin{align}
  \muplus - \mumin &= \muplus - \frac{\nplus}{\nmin} \muplus \\
  &= \frac{\nmin - \nplus}{\nmin} \muplus \\
  \therefore \muplus &= \frac{\nmin}{\nmin - \nplus} (\muplus - \mumin) \label{equation:mu}
\end{align}
に留意する.
次に,左辺の\(\Xt X\)について,
\begin{align}
  \Xt X &= \sum_{i=1}^{n}{\x_i \xt_i} \\
  &= \sum_{y_i=1}{\x_{i, +} \xt_{i, +}} + \sum_{y_i=-1}{\x_{i, -} \xt_{i, -}}
\end{align}
と表わせ,まとめて
\begin{align}
  t &\in \{+, -\} \nonumber \\
  y_t &= \begin{cases} 1 & (t=+) \\ -1 & (t=-) \end{cases} \nonumber
\end{align}
とおくと,\(\sum_{y_i = y_t}{\x_{i, t} \x_{i, t}^\top}\)について,
\begin{align}
  \widehat{\Sigma}_t &= \frac{1}{\nt} \sum_{y_i = y_t}{(\x_{i, t} - \muv_{t}) (\x_{i, t} - \muv_{t})^\top} \\
  &= \frac{1}{\nt} (\sum_{y_i = y_t}{\x_{i, t} \x_{i, t}^\top} - 2 \muv_{t} \sum_{y_i = y_t}{\x_{i, t}^\top} + \sum_{y_i = y_t}{\muv_t \muv_t^\top}) \\
  &= \frac{1}{\nt} (\sum_{y_i = y_t}{\x_{i, t} \x_{i, t}^\top} - 2 \nt \muv_{t} \muv_{t}^\top + \nt \muv_t \muv_t^\top) \\
  &= \frac{1}{\nt} (\sum_{y_i = y_t}{\x_{i, t} \x_{i, t}^\top} - \nt \muv_{t} \muv_{t}^\top)
\end{align}
より,
\begin{align}
  \sum_{y_i = y_t}{\x_{i, t} \x_{i, t}^\top} &= \nt \widehat{\Sigma}_t + \nt \muv_{t} \muv_{t}^\top
\end{align}
と表せるので,
\begin{align}
  \Xt X &= (\nplus \widehat{\Sigma}_{+} + \nplus \muplus \muplus^\top) + (\nmin \widehat{\Sigma}_{-} + \nmin \mumin \mumin^\top) \\
  &= (\nplus + \nmin) \widehat{\Sigma} + \nplus \muplus \muplus^\top + \nmin \left(\frac{\nplus}{\nmin}\right)^2 \muplus \muplus^\top \\
  &= (\nplus + \nmin) \left( \widehat{\Sigma} + \frac{\nplus}{\nmin} \muplus \muplus^\top \right)
\end{align}
となる.これらを式\eqref{equation:problem}に代入すると,
\begin{align}
  \Xt X \thv &= \Xt \y \\
  (\nplus + \nmin) \left( \widehat{\Sigma} + \frac{\nplus}{\nmin} \muplus \muplus^\top \right) \thv &= 2 \nplus \muplus \\
  \widehat{\Sigma} \thv + \frac{\nplus}{\nmin} \muplus \muplus^\top \thv &= \frac{2 \nplus}{\nplus + \nmin} \muplus \\
  \widehat{\Sigma} \thv &= \left( \frac{2 \nplus}{\nplus + \nmin} - C_{\muplus, \theta} \cdot \frac{\nplus}{\nmin} \right) \muplus \label{equation:hint} \\
  \thv &= C \cdot \widehat{\Sigma}^{-1} \muplus \\
  \thv &= C' \cdot \widehat{\Sigma}^{-1} (\muplus - \mumin) \ \ (\because 式\eqref{equation:mu})
\end{align}
よって,最小二乗最適化により得られるベクトル\(\hat{\thv}\)は,フィッシャー判別分析による境界の垂線\(\widehat{\Sigma}^{-1} (\widehat{\vector{\mu}}_{+} - \widehat{\vector{\mu}}_{-})\)の定数倍であることが示された.\par
ただし,式\eqref{equation:hint}では,任意の\(\vector{v}, \vector{\theta}\)について,ある定数\(c\)が存在し,
\begin{align}
  \vector{v} \vector{v}^\top \vector{\theta} = c \cdot \vector{v}
\end{align}
が成り立つことを用いた.


%% \bibliographystyle{junsrt}
%% \bibliography{p-report}

\end{document}

