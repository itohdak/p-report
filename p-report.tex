\documentclass[onecolumn]{preport}
\usepackage[dvipdfmx]{graphicx}
\graphicspath{{figs/}}

\usepackage{listings}

\title{1192年度 研究要旨: \\
プログレスレポートのテンプレート}
\author{1192年 源頼朝}

\begin{document}

\pagestyle{empty}
\maketitle
\thispagestyle{empty}
\sloppy

\section{はじめに}

本稿はプログレスレポートのテンプレートである\cite{Sakai}.

本稿における「、」や「。」は、\verb|make pub|を実行することで、「,」や「.」に変更される。

図は\figref{nowprinting}や\tabref{sample}として参照する.

\begin{figure}[tbh]
 \begin{center}
  \begin{minipage}{0.3\columnwidth}
   \includegraphics[width=\columnwidth]{nowprinting.eps}
   \caption{eps図の参考例}
  \end{minipage}
  \hspace{0.15\columnwidth}
  \begin{minipage}{0.3\columnwidth}
   \includegraphics[width=\columnwidth]{dj.jpg}
   \caption{jpg図の参考例}
  \end{minipage}
  \label{figure:nowprinting}
 \end{center}
\end{figure}

\begin{table}[tbh]
 \begin{center}
  \begin{tabular}{|l|r|} \hline
  A1 & B1 \\
  A2 & B2 \\ \hline
  \end{tabular}
  \caption{図の参考例}
  \label{table:sample}
 \end{center}
\end{table}

\begin{lstlisting}[basicstyle=\ttfamily\footnotesize, frame=single]
#include <iostream>
#include <stdlib.h>
#include <time.h>

#define N 100
double A[N][N], B[N][N], C[N][N];

int main(){
  /* initialize */
  srand((unsigned int)time(NULL));
  for(int i=0; i<N; i++){
    for(int j=0; j<N; j++){
      A[i][j] = 1.0 * rand() / RAND_MAX;
      B[i][j] = 1.0 * rand() / RAND_MAX;
      C[i][j] = 0.0;
    }
  }
  std::cout << "initialization finish" << std::endl;

  clock_t start = clock();
  for(int i=0; i<N; i++)
    for(int j=0; j<N; j++)
      for(int k=0; k<N; k++)
	C[i][j] += A[i][k] * B[k][j];
  clock_t end = clock();
  std::cout << 1e+3 * (double)(end-start) / CLOCKS_PER_SEC << " msec." << std::endl;

  return 0;
}
\end{lstlisting}
\begin{lstlisting}[basicstyle=\ttfamily\footnotesize, frame=single]
public class mat {
    public static void main(String[] args){
	int N = 100;
	double[][] A = new double[N][N];
	double[][] B = new double[N][N];
	double[][] C = new double[N][N];
	for(int i=0; i<N; i++){
	    for(int j=0; j<N; j++){
		A[i][j] = Math.random();
		B[i][j] = Math.random();
		C[i][j] = 0.0;
	    }
	}
	System.out.println("initialization finish");

	long start = System.currentTimeMillis();
	for(int i=0; i<N; i++)
	    for(int j=0; j<N; j++)
		for(int k=0; k<N; k++)
		    C[i][j] += A[i][k] * B[k][j];
	long end = System.currentTimeMillis();
	System.out.println((end-start) + " msecs.");

    }
}
\end{lstlisting}
\begin{lstlisting}[basicstyle=\ttfamily\footnotesize, frame=single]
import random
import time

if __name__ == '__main__':
    N = 100
    A = [[random.uniform(0.0, 1.0) for i in range(N)] for j in range(N)]
    B = [[random.uniform(0.0, 1.0) for i in range(N)] for j in range(N)]
    C = [[0.0 for i in range(N)] for j in range(N)]
    print("initialization finish")

    start = time.time()
    for i in range(N):
        for j in range(N):
            for k in range(N):
                C[i][j] += A[i][k] * B[k][j]
    end = time.time()
    print("{} msec.".format((end-start)*1e+3))
\end{lstlisting}

\section{おわりに}

\bibliographystyle{junsrt}
\bibliography{p-report}

\end{document}

